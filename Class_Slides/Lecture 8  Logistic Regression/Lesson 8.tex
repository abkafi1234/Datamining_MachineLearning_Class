\documentclass{beamer}
\usepackage{listings}
\usepackage{graphicx}
\usepackage{amsmath, amsfonts}
\usepackage{graphicx}
\usepackage{booktabs}
\usepackage{hyperref}
\usepackage{verbatim}
\usetheme{metropolis} % Clean modern look
\title{Logistic Regression}
\subtitle{}
\date{}

\begin{document}

% Title Slide
{
\setbeamertemplate{frame footer}{\href{https://github.com/abkafi1234/Datamining_MachineLearning_Class/tree/main/Class_Slides}{Report Error}}
\setbeamerfont{frame footer}{size=\tiny} 
\begin{frame}
    \titlepage
\end{frame}
}

\begin{frame}{Outline}
    \tableofcontents
\end{frame}



\section{Introduction and Theory}

\begin{frame}{What is Logistic Regression?}
\begin{itemize}
  \item A statistical model for binary classification.
  \item Predicts the probability of class membership.
  \item Output: probability, mapped to classes using sigmoid.
\end{itemize}
\end{frame}

\begin{frame}{Logistic vs Linear Regression}
\begin{itemize}
  \item \textbf{Linear Regression:} Predicts continuous outcomes.
  \item \textbf{Logistic Regression:} Predicts probability of class membership.
  \item Uses the sigmoid function:
  \[
  \sigma(z) = \frac{1}{1 + e^{-z}}, \quad z = \beta_0 + \beta_1 x_1 + \dots + \beta_n x_n
  \]
\end{itemize}
\end{frame}

\section{Types and Assumptions}

\begin{frame}{Types of Logistic Regression}
\begin{itemize}
  \item \textbf{Binary}: Two classes (e.g., spam or not spam).
  \item \textbf{Multinomial}: Three or more unordered classes.
  \item \textbf{Ordinal}: Ordered categorical outcomes.
\end{itemize}
\end{frame}

\begin{frame}{Assumptions}
\begin{itemize}
  \item Binary dependent variable.
  \item Logit is linear in predictors.
  \item No multicollinearity.
  \item Observations are independent.
  \item Large sample size preferred.
\end{itemize}
\end{frame}

\section{Mathematical Example}
\begin{frame}{Example Dataset: Fruits}
\begin{table}[h!]
\centering
\begin{tabular}{@{}cccc@{}}
\toprule
\textbf{Fruit} & \textbf{Weight (g)} & \textbf{Color Score} & \textbf{Apple? (y)} \\
\midrule
1 & 150 & 0.9 & 1 \\
2 & 180 & 0.8 & 1 \\
3 & 120 & 0.3 & 0 \\
4 & 130 & 0.4 & 0 \\
\bottomrule
\end{tabular}
\end{table}
\end{frame}

% Model Formula Slide
\begin{frame}{Logistic Regression Model}
\begin{block}{Model Equation}
We model the probability of a fruit being an apple using:
\[
z = \beta_0 + \beta_1 x_1 + \beta_2 x_2
\]
\[
\hat{y} = \sigma(z) = \frac{1}{1 + e^{-z}}
\]
\end{block}
\begin{itemize}
  \item \( x_1 \): Weight (in grams)
  \item \( x_2 \): Color score (0 = green, 1 = red)
\end{itemize}
\end{frame}

% Coefficients and New Input
\begin{frame}{Given Coefficients and New Input}
\begin{itemize}
  \item Coefficients:
  \[
  \beta_0 = -20,\quad \beta_1 = 0.1,\quad \beta_2 = 5
  \]
  \item New fruit input:
  \[
  x_1 = 160,\quad x_2 = 0.85
  \]
\end{itemize}
\end{frame}

% Math Calculation Slide
\begin{frame}{Prediction Step-by-Step}
\begin{block}{Step 1: Compute Linear Combination}
\[
z = -20 + 0.1 \cdot 160 + 5 \cdot 0.85 = -20 + 16 + 4.25 = 0.25
\]
\end{block}

\begin{block}{Step 2: Apply Sigmoid Function}
\[
\hat{y} = \frac{1}{1 + e^{-0.25}} \approx \frac{1}{1 + 0.7788} \approx 0.562
\]
\end{block}
\end{frame}

% Interpretation
\begin{frame}{Interpretation}
\begin{itemize}
  \item The model predicts a probability of \( \hat{y} = 0.562 \).
  \item With a threshold of 0.5, we classify the fruit as:
  \[
  \textbf{Apple (1)}
  \]
\end{itemize}
\end{frame}


\section{Implementation}

\begin{frame}[fragile]{Basic Implementation in Python}
\begin{itemize}
  \item Uses \texttt{scikit-learn}'s \texttt{LogisticRegression}.
\end{itemize}
\begin{block}{Example}
\begin{verbatim}
from sklearn.linear_model import LogisticRegression
model = LogisticRegression()
model.fit(X_train, y_train)
y_pred = model.predict(X_test)
\end{verbatim}
\end{block}
\end{frame}

\begin{frame}{Key Hyperparameters}
\begin{itemize}
  \item \texttt{penalty}: L1, L2, elasticnet, or none.
  \item \texttt{C}: Inverse of regularization strength.
  \item \texttt{solver}: liblinear, saga, lbfgs, etc.
  \item \texttt{class\_weight}: Deal with imbalance.
\end{itemize}
\end{frame}

\section{Model Evaluation}

\begin{frame}{Evaluation Metrics}
\begin{itemize}
  \item Accuracy, Precision, Recall, F1 Score.
  \item Confusion Matrix.
  \item ROC AUC, Log Loss, Brier Score.
\end{itemize}
\end{frame}

\begin{frame}{Probability Calibration}
\begin{itemize}
  \item Use \texttt{calibration\_curve} or \texttt{CalibratedClassifierCV}.
  \item Important in risk-sensitive applications.
\end{itemize}
\end{frame}

\section{Regularization and Feature Importance}

\begin{frame}{Regularization Techniques}
\begin{itemize}
  \item \textbf{L1 (Lasso)}: Feature selection.
  \item \textbf{L2 (Ridge)}: Shrinks large coefficients.
  \item \textbf{ElasticNet}: Mix of both.
\end{itemize}
\end{frame}

\begin{frame}{Interpreting Coefficients}
\begin{itemize}
  \item Coefficients represent change in log-odds.
  \item Odds Ratio: \( \text{OR} = e^{\beta} \)
  \item Standardize features before interpretation.
\end{itemize}
\end{frame}

\section{Advanced Topics}

\begin{frame}{Multiclass and Imbalanced Data}
\begin{itemize}
  \item Use \texttt{multi\_class='multinomial'} for >2 classes.
  \item Handle imbalance with:
  \begin{itemize}
    \item \texttt{class\_weight='balanced'}
    \item SMOTE / undersampling
  \end{itemize}
\end{itemize}
\end{frame}

\begin{frame}{Diagnostics and Feature Selection}
\begin{itemize}
  \item Use VIF to detect multicollinearity.
  \item Recursive Feature Elimination (RFE).
  \item Hosmer-Lemeshow test for goodness of fit.
\end{itemize}
\end{frame}

\section{Tools and Best Practices}

\begin{frame}{Tools and Libraries}
\begin{itemize}
  \item \textbf{Modeling:} \texttt{sklearn.linear\_model}
  \item \textbf{Metrics:} \texttt{sklearn.metrics}
  \item \textbf{Visualization:} \texttt{seaborn, matplotlib, yellowbrick}
  \item \textbf{Feature Selection:} \texttt{SelectFromModel, RFE}
\end{itemize}
\end{frame}

\begin{frame}{Best Practices}
\begin{itemize}
  \item Scale features for regularization.
  \item Handle class imbalance carefully.
  \item Evaluate both class labels and probabilities.
  \item Validate with cross-validation.
  \item Understand your coefficients.
\end{itemize}
\end{frame}
% Outline Slide


\begin{frame}[standout]
    Thank you!
\end{frame}
\end{document}
