\documentclass{beamer}
\usetheme{metropolis} % Clean modern look

\usepackage{amsmath, amsfonts}
\usepackage{booktabs}
\usepackage{hyperref}

\title{Convolutional Neural Networks}
\subtitle{Understanding CNNs from Basics}
\date{July 19, 2025}

\begin{document}

{
\setbeamertemplate{frame footer}{\href{https://github.com/abkafi1234/Datamining_MachineLearning_Class/tree/main/Class_Slides}{Report Error}}
\setbeamerfont{frame footer}{size=\tiny} 
\begin{frame}
    \titlepage
\end{frame}
}

\begin{frame}{Outline}
    \tableofcontents
\end{frame}

% --- What is CNN ---
\section{What is a CNN?}
\begin{frame}{What is a Convolutional Neural Network?}
CNNs are a class of deep learning models primarily used for image, video, and spatial data.
\begin{itemize}
    \item Automatically extract spatial features using filters
    \item Replaces manual feature extraction
    \item Efficient in capturing spatial hierarchies
\end{itemize}
\end{frame}

\begin{frame}{Why Use CNNs?}
\begin{itemize}
    \item Fully connected networks do not scale well for image data
    \item CNNs preserve spatial locality using convolution
    \item Fewer parameters, shared weights
    \item Effective in object detection, segmentation, and classification
\end{itemize}
\end{frame}

% --- Convolution Operation ---
\section{Convolution Layer}
\begin{frame}{What is Convolution?}
\textbf{Convolution} is a weighted sum of inputs in a local region.
\begin{itemize}
    \item Applies a small filter (kernel) over the input
    \item Captures local features like edges, textures
    \item Parameters are shared across spatial locations
\end{itemize}
\end{frame}

\begin{frame}{Convolution Math}
Let:
\begin{itemize}
    \item Input: \( I \in \mathbb{R}^{H \times W} \)
    \item Kernel: \( K \in \mathbb{R}^{k \times k} \)
\end{itemize}
Then, output at position \( (i, j) \) is:
\[
S(i, j) = \sum_{m=0}^{k-1} \sum_{n=0}^{k-1} I(i+m, j+n) \cdot K(m, n)
\]
\end{frame}

\begin{frame}{Output Shape Formula}
Given:
\begin{itemize}
    \item Input size: \( H_{in} \times W_{in} \)
    \item Kernel size: \( K \)
    \item Padding: \( P \)
    \item Stride: \( S \)
\end{itemize}
Then:
\[
H_{out} = \left\lfloor \frac{H_{in} + 2P - K}{S} + 1 \right\rfloor
\quad
W_{out} = \left\lfloor \frac{W_{in} + 2P - K}{S} + 1 \right\rfloor
\]
\end{frame}

\begin{frame}{Example: Output Size}
\begin{itemize}
    \item Input: \( 32 \times 32 \)
    \item Kernel: \( 5 \times 5 \)
    \item Stride: 1, Padding: 0
\end{itemize}
\[
H_{out} = \frac{32 - 5}{1} + 1 = 28,\quad
W_{out} = \frac{32 - 5}{1} + 1 = 28
\]
\textit{Output size: \( 28 \times 28 \)}
\end{frame}

% --- Padding and Stride ---
\section{Padding and Stride}
\begin{frame}{What is Padding?}
\begin{itemize}
    \item Adds border around the input
    \item Keeps output the same size (when needed)
    \item Common types: 'valid' (no padding), 'same' (pad to preserve size)
\end{itemize}
\end{frame}

\begin{frame}{What is Stride?}
\begin{itemize}
    \item Number of pixels the filter moves per step
    \item Stride > 1 reduces output size
    \item Acts like downsampling
\end{itemize}
\end{frame}

% --- Activation + Pooling ---
\section{Activation and Pooling}
\begin{frame}{Activation Function}
\textbf{ReLU (Rectified Linear Unit)} is most commonly used:
\[
f(x) = \max(0, x)
\]
\begin{itemize}
    \item Introduces non-linearity
    \item Helps prevent vanishing gradients
\end{itemize}
\end{frame}

\begin{frame}{Pooling Layers}
\textbf{Max Pooling} and \textbf{Average Pooling} reduce spatial dimensions.
\begin{itemize}
    \item Aggregates regions to smaller representation
    \item Reduces computation, controls overfitting
\end{itemize}
Example:
\begin{itemize}
    \item Input: \( 4 \times 4 \), Pool: \( 2 \times 2 \), Stride: 2
    \item Output: \( 2 \times 2 \)
\end{itemize}
\end{frame}

% --- Architecture ---
\section{CNN Architecture}
\begin{frame}{Typical CNN Architecture}
\begin{itemize}
    \item Input Layer (e.g., image)
    \item Convolution → ReLU → Pooling
    \item Repeat above layers
    \item Flatten → Fully Connected Layer(s)
    \item Output Layer (Softmax or Sigmoid)
\end{itemize}
\end{frame}

\begin{frame}{Parameter Sharing and Sparsity}
\begin{itemize}
    \item Each filter scans entire image but uses same weights
    \item Fewer parameters than fully connected layers
    \item Efficient training and less overfitting
\end{itemize}
\end{frame}

% --- Use Cases ---
\section{Use Cases}
\begin{frame}{Applications of CNNs}
\begin{itemize}
    \item Image classification (e.g., CIFAR-10, ImageNet)
    \item Object detection (YOLO, Faster R-CNN)
    \item Face recognition and biometrics
    \item Medical image analysis
    \item NLP (e.g., text classification with 1D CNNs)
\end{itemize}
\end{frame}

\begin{frame}{Benefits of CNNs}
\begin{itemize}
    \item Captures spatial features automatically
    \item Good generalization with less data
    \item State-of-the-art for image-based tasks
\end{itemize}
\end{frame}

% --- Summary ---
\begin{frame}{Key Concepts Recap}
\begin{itemize}
    \item Convolution: Local feature detection
    \item Pooling: Dimensionality reduction
    \item ReLU: Non-linearity
    \item CNNs are efficient for structured data like images
\end{itemize}
\end{frame}

\begin{frame}[standout]
    Thank you! \\
    Questions?
\end{frame}

\end{document}
