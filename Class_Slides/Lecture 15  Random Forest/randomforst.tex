\documentclass{beamer}
\usetheme{metropolis}

\usepackage{amsmath, amsfonts}
\usepackage{graphicx}
\usepackage{booktabs}
\usepackage{hyperref}

\title{Random Forest}
\subtitle{A Gentle Introduction}
\date{}

\begin{document}

% Title Slide
{
\setbeamertemplate{frame footer}{\href{https://github.com/abkafi1234/Datamining_MachineLearning_Class/tree/main/Class_Slides}{Report Error}}
\setbeamerfont{frame footer}{size=\tiny}
\begin{frame}
    \titlepage
\end{frame}
}

% Outline
\begin{frame}{Outline}
    \tableofcontents
\end{frame}

% Section 1
\section{Introduction}

\begin{frame}{What is a Random Forest?}
\begin{itemize}
    \item An ensemble of decision trees
    \item Combines predictions from multiple trees
    \item Trained on different subsets of data and features
    \item Used for classification and regression
\end{itemize}
\end{frame}

\begin{frame}{Why Use Random Forests?}
\begin{itemize}
    \item Reduces overfitting of individual decision trees
    \item Increases predictive accuracy
    \item Handles high-dimensional and missing data well
    \item Requires minimal parameter tuning
\end{itemize}
\end{frame}

% Section 2
\section{How It Works}

\begin{frame}{Algorithm Overview}
\begin{enumerate}
    \item Draw $N$ bootstrap samples from training data
    \item Train a decision tree on each sample:
    \begin{itemize}
        \item At each split, consider only a random subset of features
    \end{itemize}
    \item Aggregate predictions:
    \begin{itemize}
        \item Majority vote (classification)
        \item Average (regression)
    \end{itemize}
\end{enumerate}
\end{frame}

\begin{frame}{Voting and Averaging}
\begin{itemize}
    \item \textbf{Classification}:
    \[
    \hat{y} = \text{majority vote of all trees}
    \]
    \item \textbf{Regression}:
    \[
    \hat{y} = \frac{1}{T} \sum_{t=1}^{T} f_t(x)
    \]
    \item Intuition: reduces variance, like averaging noisy opinions
\end{itemize}
\end{frame}

% Section 3
\section{Features and Parameters}

\begin{frame}{Key Hyperparameters}
\begin{itemize}
    \item \texttt{n\_estimators}: Number of trees
    \item \texttt{max\_depth}: Max depth of each tree
    \item \texttt{max\_features}: Number of features considered at each split
    \item \texttt{min\_samples\_split}: Minimum samples required to split a node
    \item \texttt{bootstrap}: Whether to use bootstrap samples
\end{itemize}
\end{frame}

\begin{frame}{Out-of-Bag (OOB) Error Estimate}
\begin{itemize}
    \item Each tree is trained on a bootstrap sample
    \item About 1/3 of data is left out ("out-of-bag")
    \item These OOB samples are used to:
    \begin{itemize}
        \item Estimate generalization error
        \item Avoid cross-validation
    \end{itemize}
\end{itemize}
\end{frame}

% Section 4
\section{Pros and Cons}

\begin{frame}{Advantages of Random Forest}
\begin{itemize}
    \item High accuracy with minimal tuning
    \item Works well on many data types and tasks
    \item Robust to outliers and noise
    \item Handles large datasets and features efficiently
    \item Feature importance analysis available
\end{itemize}
\end{frame}

\begin{frame}{Limitations}
\begin{itemize}
    \item Slower for large number of trees or very deep trees
    \item Less interpretable than a single decision tree
    \item May overfit if trees are too deep and data is noisy
    \item Not ideal for extrapolation tasks (in regression)
\end{itemize}
\end{frame}

% Section 5
\section{Applications \& Tools}

\begin{frame}{Common Applications}
\begin{itemize}
    \item Medical diagnosis
    \item Credit scoring and fraud detection
    \item Recommendation systems
    \item Image and text classification
\end{itemize}
\end{frame}

\begin{frame}{Tools and Libraries}
\begin{itemize}
    \item \textbf{Scikit-learn}: \texttt{RandomForestClassifier}, \texttt{RandomForestRegressor}
    \item \textbf{Spark MLlib}, \textbf{H2O.ai}, \textbf{Weka}
    \item Deep integration in data science workflows
\end{itemize}
\end{frame}

% Summary
\section{Summary}

\begin{frame}{Key Takeaways}
\begin{itemize}
    \item Random Forests = Ensemble of Decision Trees
    \item Based on bagging + random feature selection
    \item Reduces overfitting, improves accuracy
    \item Useful, robust, and easy to use
\end{itemize}
\end{frame}

\begin{frame}[standout]
    Thank you!
\end{frame}

\end{document}
