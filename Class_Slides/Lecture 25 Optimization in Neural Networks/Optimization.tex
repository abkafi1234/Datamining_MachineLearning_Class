\documentclass{beamer}
\usetheme{metropolis} % Clean modern look

\usepackage{amsmath, amsfonts}
\usepackage{booktabs}
\usepackage{hyperref}

\title{Optimization in Neural Networks}
\subtitle{How do neural networks learn?}
\date{July 19, 2025}

\begin{document}

{
\setbeamertemplate{frame footer}{\href{https://github.com/abkafi1234/Datamining_MachineLearning_Class/tree/main/Class_Slides}{Report Error}}
\setbeamerfont{frame footer}{size=\tiny} 
\begin{frame}
    \titlepage
\end{frame}
}

\begin{frame}{Outline}
    \tableofcontents
\end{frame}

% --- Why Optimization ---
\section{Why Optimization?}
\begin{frame}{Why Optimization in Neural Networks?}
Optimization is the process of minimizing a loss function by adjusting the model's parameters.
\begin{itemize}
    \item Allows neural networks to learn patterns from data
    \item Uses gradients to guide parameter updates
    \item Goal: Find parameter values that minimize the loss function
\end{itemize}
\end{frame}

\begin{frame}{Loss Function Examples}
\textbf{Mean Squared Error (MSE):}
\[
\text{MSE} = \frac{1}{n} \sum_{i=1}^n (y_i - \hat{y}_i)^2
\]

\textbf{Cross-Entropy Loss:}
\[
\text{CE} = -\sum_i y_i \log(\hat{y}_i)
\]
\begin{itemize}
    \item MSE is used in regression
    \item Cross-entropy is used in classification
\end{itemize}
\end{frame}

% --- Gradient Descent Basics ---
\section{Gradient Descent}
\begin{frame}{Gradient Descent: Core Idea}
Gradient Descent iteratively updates parameters in the direction of the negative gradient.
\[
w := w - \eta \cdot \nabla L(w)
\]
\begin{itemize}
    \item \( \eta \): Learning rate
    \item \( \nabla L(w) \): Gradient of loss with respect to parameters
\end{itemize}
\end{frame}

\begin{frame}{Types of Gradient Descent}
\begin{itemize}
    \item \textbf{Batch Gradient Descent}: Uses entire dataset
    \item \textbf{Stochastic Gradient Descent (SGD)}: One sample at a time
    \item \textbf{Mini-batch Gradient Descent}: Uses a subset (e.g., 32 samples)
\end{itemize}
\end{frame}

\begin{frame}{GD Variant Comparison}
\footnotesize
\begin{tabular}{@{}lll@{}}
\toprule
\textbf{Type} & \textbf{Pros} & \textbf{Cons} \\
\midrule
Batch & Stable convergence & Computationally expensive \\
SGD & Fast, online learning & Noisy updates \\
Mini-batch & Balanced speed & Requires tuning batch size \\
\bottomrule
\end{tabular}
\end{frame}

% --- Optimizer Algorithms ---
\section{Popular Optimizers}
\begin{frame}{SGD with Momentum}
Momentum adds velocity to help accelerate convergence:
\[
v_t = \beta v_{t-1} + \eta \nabla L(w), \quad w := w - v_t
\]
\begin{itemize}
    \item \( \beta \): Momentum coefficient (e.g., 0.9)
    \item Helps escape saddle points and oscillations
\end{itemize}
\end{frame}

\begin{frame}{RMSProp}
Adapts learning rate for each parameter:
\[
E[g^2]_t = \gamma E[g^2]_{t-1} + (1-\gamma)g_t^2
\]
\[
w := w - \frac{\eta}{\sqrt{E[g^2]_t + \epsilon}} \cdot g_t
\]
\begin{itemize}
    \item Good for RNNs
    \item Controls exploding gradients
\end{itemize}
\end{frame}

\begin{frame}{Adam Optimizer}
Combines momentum and RMSProp:
\[
m_t = \beta_1 m_{t-1} + (1 - \beta_1) g_t
\]
\[
v_t = \beta_2 v_{t-1} + (1 - \beta_2) g_t^2
\]
\[
\hat{m}_t = \frac{m_t}{1 - \beta_1^t},\quad
\hat{v}_t = \frac{v_t}{1 - \beta_2^t}
\]
\[
w := w - \frac{\eta}{\sqrt{\hat{v}_t} + \epsilon} \cdot \hat{m}_t
\]
\end{frame}

\begin{frame}{Optimizer Comparison}
\footnotesize
\begin{tabular}{@{}llll@{}}
\toprule
\textbf{Optimizer} & \textbf{Pros} & \textbf{Cons} & \textbf{Use Case} \\
\midrule
SGD & Simple, low memory & Slow convergence & General baseline \\
Momentum & Escapes local minima & Needs tuning & Deep nets \\
RMSProp & Stable updates & Sensitive to hyperparams & RNNs \\
Adam & Fast, adaptive & May overfit & Most models \\
\bottomrule
\end{tabular}
\end{frame}

% --- Challenges ---
\section{Challenges in Optimization}
\begin{frame}{Optimization Challenges}
\begin{itemize}
    \item \textbf{Saddle Points} slow down learning
    \item \textbf{Vanishing/Exploding Gradients} in deep nets
    \item \textbf{Overfitting}: Excessive optimization on training data
\end{itemize}
\end{frame}

\begin{frame}{Learning Rate Schedules}
\begin{itemize}
    \item \textbf{Step decay}: Reduce \( \eta \) after fixed steps
    \item \textbf{Exponential decay}: Multiply \( \eta \) each epoch
    \item \textbf{Reduce on plateau}: Monitor validation loss
\end{itemize}
\end{frame}

% --- Best Practices ---
\section{Best Practices}
\begin{frame}{Tips \& Best Practices}
\begin{itemize}
    \item Start with Adam, tune learning rate
    \item Use mini-batches and validation sets
    \item Monitor training and validation loss
    \item Use TensorBoard or similar tools
\end{itemize}
\end{frame}

% --- Recap ---
\begin{frame}{Key Takeaways}
\begin{itemize}
    \item Optimization drives learning in neural networks
    \item Gradient descent is foundational
    \item Adam optimizer is robust and widely used
    \item Always validate, tune, and monitor
\end{itemize}
\end{frame}

\begin{frame}[standout]
    Thank you! \\
    Questions?
\end{frame}

\end{document}
