\documentclass{beamer}
\usetheme{metropolis} % Clean modern look

\usepackage{amsmath, amsfonts}
\usepackage{graphicx}
\usepackage{booktabs}
\usepackage{hyperref}

\title{Feature Selection and Engineering}
\subtitle{}
\date{}

\begin{document}

% Title Slide
{
\setbeamertemplate{frame footer}{\href{https://github.com/abkafi1234/Datamining_MachineLearning_Class/tree/main/Class_Slides}{Report Error}}
\setbeamerfont{frame footer}{size=\tiny} 
\begin{frame}
    \titlepage
\end{frame}
}

% Outline Slide
\begin{frame}{Outline}
    \tableofcontents
\end{frame}

% Section: Introduction
\section{Introduction}

\begin{frame}{Introduction}
  \bigskip
  \textbf{What is Feature Engineering?}\\ 
  Feature Engineering is the process of using domain knowledge to extract features from raw data that make machine learning algorithms work better.
  \bigskip

  \textbf{What is Feature Selection?}\\  
  Feature Selection is the process of selecting a subset of relevant features for use in model construction. It helps reduce overfitting, improves accuracy, and decreases training time.
\end{frame}

% Section: Feature Engineering Techniques
\section{Feature Engineering Techniques}

% Handling Missing Values
\begin{frame}{Handling Missing Values}
  \textit{Replacing or flagging missing data to ensure model integrity.}  

  \textbf{Examples:}  
  \begin{itemize}
    \item \textbf{Imputation}: Fill missing values using Mean/Median/Mode.
    \item \textbf{Interpolation}: Estimate missing values based on nearby data points.
    \item \textbf{Model-based}: Use KNN or MICE to predict missing values.
    \item \textbf{Flag Missingness}: Create a binary indicator variable for missing data.
  \end{itemize}
\end{frame}

% Encoding Categorical Variables
\begin{frame}{Encoding Categorical Variables}
  \textit{Convert categorical data into numerical format for ML models.}  

  \textbf{Examples:}  
  \begin{itemize}
    \item Label Encoding (Ordinal data)
    \item One-Hot Encoding (Nominal data)
    \item Target Encoding (Category replaced with target mean)
    \item Frequency Encoding (Category replaced with frequency count)
  \end{itemize}
\end{frame}

% Scaling and Normalization
\begin{frame}{Scaling and Normalization}
  \textit{Adjust feature values to improve model efficiency and stability.}  

  \textbf{Examples:}  
  \begin{itemize}
    \item Min-Max Scaling (Rescale values between 0 and 1)
    \item Standardization (Z-score normalization)
    \item Robust Scaling (Handles outliers effectively)
  \end{itemize}
\end{frame}

% Date and Time Features
\begin{frame}{Date and Time Features}
  \textit{Extract meaningful information from timestamps.}  

  \textbf{Examples:}  
  \begin{itemize}
    \item Extract year, month, weekday, hour
    \item Compute duration (e.g., time since last purchase)
  \end{itemize}
\end{frame}

% Text Feature Engineering
\begin{frame}{Text Feature Engineering}
  \textit{Convert text into numerical representations for ML models.}  

  \textbf{Examples:}  
  \begin{itemize}
    \item TF-IDF, Bag of Words
    \item N-grams for sequence analysis
    \item Word Embeddings (Word2Vec, GloVe, BERT)
  \end{itemize}
\end{frame}

% Polynomial and Interaction Features
\begin{frame}{Polynomial and Interaction Features}
  \textit{Enhance feature complexity for non-linear relationships.}  

  \textbf{Examples:}  
  \begin{itemize}
    \item Use PolynomialFeatures from scikit-learn
    \item Multiply or add features for richer relationships
  \end{itemize}
\end{frame}

% Aggregation and GroupBy Features
\begin{frame}{Aggregation and GroupBy Features}
  \textit{Summarize data points based on logical groups.}  

  \textbf{Examples:}  
  \begin{itemize}
    \item Compute average purchase per user
    \item Count-based features (e.g., number of transactions)
  \end{itemize}
\end{frame}

% Domain-Specific Features
\begin{frame}{Domain-Specific Features}
  \textit{Custom features designed from industry knowledge.}  

  \textbf{Examples:}  
  \begin{itemize}
    \item BMI calculated from weight and height
    \item Risk scores in fraud detection
  \end{itemize}
\end{frame}


\section{Feature Selection Techniques}

% 1. Filter Methods
\begin{frame}{Filter Methods}
    \textit{Feature selection based on statistical measures independent of the model.}

    \textbf{Examples:}  
    \begin{itemize}
        \item \textbf{Correlation Thresholding}: Remove highly correlated features (e.g., remove one of "height" and "weight" if correlation > 0.9).
        \item \textbf{Chi-Square Test}: Assess categorical features (e.g., selecting significant features for predicting customer churn).
        \item \textbf{ANOVA F-test}: Evaluate numerical features against categorical targets (e.g., testing if different income levels affect purchase decisions).
        \item \textbf{Mutual Information}: Measure dependence between feature and target (e.g., selecting words that strongly predict spam emails).
    \end{itemize}
\end{frame}

% 2. Wrapper Methods
\begin{frame}{Wrapper Methods}
    \textit{Feature selection performed by iteratively testing subsets using a model.}

    \textbf{Examples:}  
    \begin{itemize}
        \item \textbf{Recursive Feature Elimination (RFE)}: Remove least important features step-by-step (e.g., selecting the top 10 features for predicting house prices).
        \item \textbf{Sequential Feature Selection (SFS)}: Add/remove features based on model accuracy (e.g., improving a fraud detection model by iteratively selecting important transaction features).
        \item \textbf{Forward Selection}: Start with no features and add one at a time (e.g., selecting the best combination of factors affecting customer retention).
        % \item \textbf{Backward Elimination}: Start with all features and remove the least useful ones (e.g., pruning irrelevant financial indicators for credit risk prediction).
    \end{itemize}
\end{frame}

% 3. Embedded Methods
\begin{frame}{Embedded Methods}
    \textit{Feature selection performed automatically during model training.}

    \textbf{Examples:}  
    \begin{itemize}
        \item \textbf{Lasso Regression (L1 Regularization)}: Shrinks less important feature weights to zero (e.g., selecting the most relevant features for predicting house prices).
        \item \textbf{Ridge Regression (L2 Regularization)}: Reduces magnitude but does not eliminate features (e.g., ensuring smooth feature importance ranking in medical diagnoses).
        \item \textbf{Tree-based Feature Importance}: Rank features using models like Random Forest or XGBoost (e.g., identifying the top predictors for customer churn).
    \end{itemize}
\end{frame}

\begin{frame}{Curse of Dimensionality}
  \textbf{Curse of Dimensionality}: As the number of features increases, the volume of the feature space increases exponentially, making it harder to find patterns.\\
\end{frame}

% 4. Dimensionality Reduction
\begin{frame}{Dimensionality Reduction}
  
    \textit{\textbf{Dimensionality Reduction Techniques} Reduce feature space while preserving the most relevant information.}\\
    \textbf{Examples:}  
    \begin{itemize}
        \item \textbf{Principal Component Analysis (PCA)}: Transform correlated features into independent components (e.g., reducing hundreds of financial metrics into principal components for stock price forecasting).
        \item \textbf{Linear Discriminant Analysis (LDA)}: Maximizes class separability while reducing dimensions (e.g., distinguishing between cancer vs. non-cancer patients based on genetic markers).
        \item \textbf{t-SNE / UMAP}: Useful for data visualization (e.g., clustering customers based on behavioral similarities).
    \end{itemize}
\end{frame}

\section{Looks Good on Paper: Why Your Model Fails in the Real World }


\begin{frame}{Data Leakage}

\textbf{Definition:} When information from outside the training dataset leaks into the model training process.
\textit{\textbf{"The Devil in Disguise"}}
\begin{itemize}
    \item \textbf{Target Leakage:} Using labels or future data in training
    \item \textbf{Train-Test Contamination:} Preprocessing before data split
    \item \textbf{Temporal Leakage:} Using future time data to predict the present
\end{itemize}

\textbf{Prevention:}
\begin{itemize}
    \item Split data before any preprocessing
    \item Avoid features unavailable at inference
    \item Use time-aware splitting for time series
\end{itemize}
\end{frame}

% Data Augmentation
\begin{frame}{Data Augmentation}
\textbf{Purpose:} Increase dataset size/diversity to improve generalization.

\begin{block}{Common Techniques}
\begin{itemize}
    \item \textbf{Images:} Flip, rotate, crop, noise
    \item \textbf{Text:} Synonym replacement, back-translation
    \item \textbf{Time Series:} Jittering, slicing
\end{itemize}
\end{block}

\textbf{Cautions:}
\begin{itemize}
    \item Don’t change labels via augmentation
    \item Apply only to training data
\end{itemize}
\end{frame}

% Pitfalls
\begin{frame}{Common Pitfalls in ML}
\begin{tabular}{ll}
\toprule
\textbf{Pitfall} & \textbf{Solution} \\
\midrule
Overfitting & Regularization, cross-validation \\
Underfitting & More features or complex model \\
Imbalanced data & SMOTE, class weights \\
Data leakage & Proper split before processing \\
Incorrect preprocessing & Standardize after splitting \\
\bottomrule
\end{tabular}
\end{frame}

% Cross Validation
\begin{frame}{Cross-Validation}
\textbf{Goal:} Evaluate model stability using resampling.\textit{\textbf{"On Training set only"}}

\begin{block}{Types of Cross-Validation}
\begin{itemize}
    \item \textbf{K-Fold}: Generic, good default
    \item \textbf{Stratified K-Fold}: Maintains class balance
    \item \textbf{TimeSeriesSplit}: For time-dependent data
    \item \textbf{Group K-Fold}: Prevents group overlap
\end{itemize}
\end{block}

\textbf{Tips:}
\begin{itemize}
    \item Use nested CV for model + hyperparameter selection
    \item Always validate on \textbf{unseen data}
\end{itemize}
\end{frame}

% Summary
\begin{frame}{Summary}
\begin{itemize}
    \item Data leakage can ruin model performance — always split first!
    \item Use augmentation wisely — only on training data
    \item Avoid common ML mistakes: overfitting, imbalance, poor validation
    \item Cross-validation is key for robust evaluation
\end{itemize}

\centering
\textbf{Practice carefully, validate robustly, deploy confidently!}
\end{frame}


\begin{frame}[standout]
    Thank you!
\end{frame}
\end{document}
