\documentclass{beamer}
\usetheme{metropolis} % Clean modern look

\usepackage{amsmath, amsfonts}
\usepackage{graphicx}
\usepackage{booktabs}
\usepackage{hyperref}

\title{Performance Analysis in Machine Learning}
\subtitle{}
\date{}

\begin{document}

% Title Slide
{
\setbeamertemplate{frame footer}{\href{https://github.com/abkafi1234/Datamining_MachineLearning_Class/tree/main/Class_Slides}{Report Error}}
\setbeamerfont{frame footer}{size=\tiny} 
\begin{frame}
    \titlepage
\end{frame}
}

% Outline Slide
\begin{frame}{Outline}
    \tableofcontents
\end{frame}

% Slide 1: Introduction
\begin{frame}{What is SVM?}
    \begin{itemize}
        \item A supervised learning algorithm for classification and regression.
        \item Finds an optimal hyperplane to separate data into categories.
        \item Uses support vectors—critical data points affecting the boundary.
    \end{itemize}
\end{frame}

% Slide 2: How SVM Works
\begin{frame}{How SVM Works}
    \begin{enumerate}
        \item Data points are plotted in an n-dimensional space.
        \item A hyperplane is found that maximizes separation margin.
        \item If non-linearly separable, kernel trick is applied.
    \end{enumerate}
\end{frame}

% Slide 3: Types of SVM
\begin{frame}{Types of SVM}
    \begin{itemize}
        \item \textbf{Linear SVM}: Works when data is separable by a straight line.
        \item \textbf{Non-Linear SVM}: Uses kernel methods to project data into higher dimensions.
    \end{itemize}
\end{frame}

% Slide 4: Kernel Functions
\begin{frame}{Common Kernel Functions}
    \begin{table}[]
        \centering
        \begin{tabular}{|l|l|}
            \hline
            \textbf{Kernel Type} & \textbf{Equation} \\
            \hline
            Linear & \( K(x, y) = x^T y \) \\
            Polynomial & \( K(x, y) = (x^T y + c)^d \) \\
            RBF & \( K(x, y) = \exp(-\gamma \|x - y\|^2) \) \\
            Sigmoid & \( K(x, y) = \tanh(\alpha x^T y + c) \) \\
            \hline
        \end{tabular}
    \end{table}
\end{frame}

% Slide 5: Python Implementation
\begin{frame}[fragile]{Python Implementation of SVM}
    \begin{block}{Example Code}
        \small 
        \begin{verbatim}
from sklearn import datasets
from sklearn.model_selection import train_test_split
from sklearn.svm import SVC
from sklearn.metrics import accuracy_score

data = datasets.load_iris()
X, y = data.data, data.target
X_train, X_test, y_train, y_test = train_test_split(X, y, test_size=0.2)

svm_model = SVC(kernel='rbf', C=1.0, gamma='scale')
svm_model.fit(X_train, y_train)

accuracy = accuracy_score(y_test, svm_model.predict(X_test))
print(f"Accuracy: {accuracy:.2f}")
        \end{verbatim}
    \end{block}
\end{frame}

% Slide 6: Pros & Cons
\begin{frame}{Advantages \& Disadvantages}
    \begin{columns}
        \column{0.5\textwidth}
        \textbf{Advantages}
        \begin{itemize}
            \item Works well with high-dimensional data.
            \item Handles cases where features outnumber samples.
            \item Effective for complex classification tasks.
        \end{itemize}
        
        \column{0.5\textwidth}
        \textbf{Disadvantages}
        \begin{itemize}
            \item Computationally expensive for large datasets.
            \item Requires careful tuning of parameters (C, gamma).
            \item Sensitive to noisy data.
        \end{itemize}
    \end{columns}
\end{frame}

% Slide 7: Applications
\begin{frame}{Applications of SVM}
    \begin{itemize}
        \item Image classification (e.g., face recognition).
        \item Bioinformatics (e.g., cancer detection).
        \item Text categorization (e.g., spam filtering).
        \item Financial predictions (e.g., stock market trends).
    \end{itemize}
\end{frame}


% Section: mathematical example of SVM
\section{Mathematical Example of SVM}
% Frame 1: Dataset
\begin{frame}{Dataset}
We are given a binary classification problem:

\textbf{Positive Class (+1):}
\begin{itemize}
    \item $(3, 1)$
    \item $(3, -1)$
    \item $(6, 1)$
    \item $(6, -1)$
\end{itemize}

\textbf{Negative Class (-1):}
\begin{itemize}
    \item $(1, 0)$
    \item $(0, 1)$
    \item $(0, -1)$
    \item $(-1, 0)$
\end{itemize}
\end{frame}

% Frame 2: Augmentation
\begin{frame}{Augmented Vectors}
To include the bias in the weight vector, augment the inputs:

\[
\tilde{x}_i = [x_{i1}, x_{i2}, 1]
\]

Examples:
\begin{align*}
(3, 1) &\rightarrow (3, 1, 1) \\
(0, -1) &\rightarrow (0, -1, 1)
\end{align*}
\end{frame}

% Frame 3: Chosen Support Vectors
\begin{frame}{Simplified Problem (3 Support Vectors)}
Choose 3 support vectors to simplify the system:

\[
\begin{aligned}
\tilde{x}_1 &= (3, 1, 1), & y_1 &= +1 \\
\tilde{x}_2 &= (3, -1, 1), & y_2 &= +1 \\
\tilde{x}_3 &= (1, 0, 1), & y_3 &= -1
\end{aligned}
\]
\end{frame}

% Frame 4: Dual Equations
\begin{frame}{Dual Form Equations}
We solve:
\[
\sum_j \alpha_j y_j (\tilde{x}_j \cdot \tilde{x}_i) = y_i
\]

Resulting system:
\[
\begin{aligned}
\frac{27}{4} \alpha_1 + \frac{25}{4} \alpha_2 - 4 \alpha_3 &= 1 \\
\frac{25}{4} \alpha_1 + \frac{27}{4} \alpha_2 - 4 \alpha_3 &= 1 \\
-4 \alpha_1 - 4 \alpha_2 + 2 \alpha_3 &= -1
\end{aligned}
\]
\end{frame}

% Frame 5: Solving for Alpha
\begin{frame}{Solution for Alphas}
Solving the system:

\[
\boxed{
\alpha_1 = \frac{3}{4}, \quad
\alpha_2 = \frac{3}{4}, \quad
\alpha_3 = \frac{7}{2}
}
\]
\end{frame}

% Frame 6: Compute Weight Vector
\begin{frame}{Compute Weight Vector}
\[
\tilde{w} = \sum_{i=1}^{3} \alpha_i y_i \tilde{x}_i
\]

\[
\tilde{w} = \frac{3}{4}(3, 1, 1) + \frac{3}{4}(3, -1, 1) - \frac{7}{2}(1, 0, 1)
= (1, 0, -2)
\]

\[
\Rightarrow w = (1, 0), \quad b = -2
\]
\end{frame}

% Frame 7: Decision Boundary
\begin{frame}{Decision Boundary}
SVM Classifier:

\[
f(x) = w^T x + b = x_1 - 2
\]

Decision boundary:
\[
\boxed{x_1 = 2}
\]

A vertical line separating the two classes.
\end{frame}

\begin{frame}[standout]
    Thank you!
\end{frame}
\end{document}
