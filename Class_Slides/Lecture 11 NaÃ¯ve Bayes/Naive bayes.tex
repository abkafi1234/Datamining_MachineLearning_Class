\documentclass{beamer}
\usetheme{metropolis} % Clean modern look

\usepackage{amsmath, amsfonts}
\usepackage{graphicx}
\usepackage{booktabs}
\usepackage{hyperref}

\title{Naive Bayes}
\subtitle{}
\date{}

\begin{document}
% Title Slide
{
\setbeamertemplate{frame footer}{\href{https://github.com/abkafi1234/Datamining_MachineLearning_Class/tree/main/Class_Slides}{Report Error}}
\setbeamerfont{frame footer}{size=\tiny} 
\begin{frame}
    \titlepage
\end{frame}
}
% Outline Slide
\begin{frame}{Outline}
    \tableofcontents
\end{frame}


\section{Introduction to Naive Bayes}
\begin{frame}{What is Naive Bayes?}
  \begin{itemize}
    \item A probabilistic classification algorithm based on Bayes’ Theorem.
    \item Called "naive" due to the strong (naive) assumption of feature independence.
    \item Performs well in many practical situations, especially text classification.
  \end{itemize}
\end{frame}

% Slide 2 - Bayes' Theorem
\begin{frame}{Bayes’ Theorem}
  \[
  P(C|X) = \frac{P(X|C) \cdot P(C)}{P(X)}
  \]
  \begin{itemize}
    \item $P(C|X)$: Posterior probability of class $C$ given data $X$
    \item $P(X|C)$: Likelihood of data $X$ given class $C$
    \item $P(C)$: Prior probability of class $C$
    \item $P(X)$: Evidence (normalizing constant)
  \end{itemize}
  In classification:
  \[
  P(C|X) \propto P(X|C) \cdot P(C)
  \]
\end{frame}

% Slide 3 - Naive Assumption
\begin{frame}{Naive Independence Assumption}
  \[
  P(X|C) = \prod_{i=1}^{n} P(x_i|C)
  \]
  \begin{itemize}
    \item Assumes all features $x_i$ are conditionally independent given class $C$.
    \item Simplifies computation dramatically.
  \end{itemize}
\end{frame}

% Slide 4 - Types of Naive Bayes
\begin{frame}{Types of Naive Bayes}
  \begin{itemize}
    \item \textbf{Gaussian NB:} Continuous data, assumes normal distribution.
    \item \textbf{Multinomial NB:} Discrete counts, great for text data.
    \item \textbf{Bernoulli NB:} Binary features (0/1), presence/absence.
  \end{itemize}
\end{frame}

% Slide 5 - Gaussian Formula
\begin{frame}{Gaussian Naive Bayes}
  For continuous features:
  \[
  P(x_i | C) = \frac{1}{\sqrt{2\pi\sigma_C^2}} \exp\left(-\frac{(x_i - \mu_C)^2}{2\sigma_C^2}\right)
  \]
  \begin{itemize}
    \item $\mu_C$: Mean of feature for class $C$
    \item $\sigma_C^2$: Variance of feature for class $C$
  \end{itemize}
\end{frame}

% Slide 6 - Example Use Case
\begin{frame}{Example: Spam Detection}
  \begin{itemize}
    \item Features: Presence of words in email
    \item Classes: Spam vs Not Spam
    \item Learn:
    \begin{itemize}
      \item $P(\text{spam})$, $P(\text{not spam})$
      \item $P(\text{word}|\text{spam})$, etc.
    \end{itemize}
    \item Predict using Bayes Rule
  \end{itemize}
\end{frame}

% Slide 7 - Pros and Cons
\begin{frame}{Pros and Cons}
  \textbf{Pros:}
  \begin{itemize}
    \item Simple, fast, efficient with high-dimensional data
    \item Requires small training data
  \end{itemize}
  \textbf{Cons:}
  \begin{itemize}
    \item Assumes feature independence (often unrealistic)
    \item Sensitive to correlated features
  \end{itemize}
\end{frame}

% Slide 8 - Python Example
\begin{frame}[fragile]{Python Example}
    \small
\begin{verbatim}
from sklearn.naive_bayes import GaussianNB
from sklearn.datasets import load_iris
from sklearn.model_selection import train_test_split

X, y = load_iris(return_X_y=True)
X_train, X_test, y_train, y_test = train_test_split(X, y)

model = GaussianNB()
model.fit(X_train, y_train)
y_pred = model.predict(X_test)
\end{verbatim}
\end{frame}

% Slide 9 - Summary
\begin{frame}{Summary}
  \begin{itemize}
    \item Naive Bayes is powerful despite its simplicity.
    \item Useful in classification tasks, especially with text data.
    \item Works best with large, clean, independent features.
  \end{itemize}
\end{frame}

\section{Math Example}
\begin{frame}{Dataset}
\begin{table}[]
\centering
\begin{tabular}{cccc}
\toprule
\textbf{Day} & \textbf{Outlook} & \textbf{Windy} & \textbf{Rain?} \\
\midrule
1 & Sunny    & False & No  \\
2 & Sunny    & True  & No  \\
3 & Overcast & False & Yes \\
4 & Rainy    & False & Yes \\
5 & Rainy    & True  & No  \\
6 & Overcast & True  & Yes \\
7 & Sunny    & False & No  \\
\bottomrule
\end{tabular}
\end{table}
\end{frame}

% Task Slide
\begin{frame}{Prediction Task}
Given a new day with:
\begin{itemize}
  \item Outlook = Rainy
  \item Windy = False
\end{itemize}
Predict whether it will \textbf{Rain} using Naive Bayes with Laplace Smoothing.
\end{frame}

% Priors
\begin{frame}{Step 1: Compute Class Priors}
Total samples: 7 \\
Yes = 3, No = 4, Number of classes = 2

\[
P(\text{Yes}) = \frac{3 + 1}{7 + 2} = \frac{4}{9} \quad 
P(\text{No}) = \frac{4 + 1}{7 + 2} = \frac{5}{9}
\]
\end{frame}

% Likelihoods
\begin{frame}{Step 2: Compute Likelihoods with Laplace Smoothing}
\textbf{For Class = Yes}
\[
P(\text{Outlook=Rainy | Yes}) = \frac{1 + 1}{3 + 3} = \frac{2}{6}
\]
\[
P(\text{Windy=False | Yes}) = \frac{2 + 1}{3 + 2} = \frac{3}{5}
\]

\textbf{For Class = No}
\[
P(\text{Outlook=Rainy | No}) = \frac{1 + 1}{4 + 3} = \frac{2}{7}
\]
\[
P(\text{Windy=False | No}) = \frac{2 + 1}{4 + 2} = \frac{3}{6}
\]
\end{frame}

% Laplace Formula Explanation
\begin{frame}{Laplace Smoothing Formula}
\[
P(x_i | C) = \frac{\text{count}(x_i, C) + 1}{\text{count}(C) + V}
\]

\textbf{Where:}
\begin{itemize}
  \item $x_i$ = feature value (e.g., Rainy, False)
  \item $C$ = class label (Yes or No)
  \item $\text{count}(x_i, C)$ = number of times $x_i$ appears in class $C$
  \item $\text{count}(C)$ = total samples in class $C$
  \item $V$ = number of possible values the feature can take
\end{itemize}
\end{frame}

% Log-likelihood calculation
\begin{frame}{Step 3: Compute Log Probabilities}
\textbf{Log Posterior (Yes):}
\[
\log\left(\frac{4}{9}\right) + \log\left(\frac{2}{6}\right) + \log\left(\frac{3}{5}\right) \approx -2.420
\]

\textbf{Log Posterior (No):}
\[
\log\left(\frac{5}{9}\right) + \log\left(\frac{2}{7}\right) + \log\left(\frac{3}{6}\right) \approx -2.534
\]
\end{frame}

% Final Prediction
\begin{frame}{Final Prediction}
\[
\log P(\text{Yes} | X) = -2.42 > \log P(\text{No} | X) = -2.53
\]

\begin{block}{Conclusion}
\textbf{Prediction:} It Will Rain (Yes)
\end{block}
\end{frame}



\begin{frame}[standout]
    Thank you!
\end{frame}
\end{document}
