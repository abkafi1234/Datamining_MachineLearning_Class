\documentclass{beamer}
\usetheme{metropolis} % Clean modern look

\usepackage{amsmath, amsfonts}
\usepackage{graphicx}
\usepackage{booktabs}
\usepackage{hyperref}

\title{Ensemble Learning Theory}
\subtitle{}
\date{}

\begin{document}

% Title Slide
{
\setbeamertemplate{frame footer}{\href{https://github.com/abkafi1234/Datamining_MachineLearning_Class/tree/main/Class_Slides}{Report Error}}
\setbeamerfont{frame footer}{size=\tiny} 
\begin{frame}
    \titlepage
\end{frame}
}

% Outline Slide
\begin{frame}{Outline}
    \tableofcontents
\end{frame}


\begin{frame}{Introduction to Ensemble Learning}
\begin{itemize}
    \item \textbf{Ensemble Learning} combines multiple models to improve performance.
    \item Motivation:
    \begin{itemize}
        \item Reduce overfitting
        \item Improve generalization
        \item Handle complex tasks
    \end{itemize}
    \item Widely used in practice (e.g., Random Forest, XGBoost, Kaggle solutions)
\end{itemize}
\end{frame}
% -- Explain real-world motivation and the general idea.

% Slide 2
\begin{frame}{Bias-Variance Decomposition}
\begin{itemize}
    \item Error = Bias$^2$ + Variance + Irreducible Error
    \item Ensembles reduce variance (Bagging), sometimes bias (Boosting)
    \item Averages out noise and instability
\end{itemize}
\[
\text{Error} = \underbrace{\text{Bias}^2}_{\text{underfitting}} + \underbrace{\text{Variance}}_{\text{overfitting}} + \text{Noise}
\]
\end{frame}
% -- Describe bias vs variance, and how ensembles help balance them.

% Slide 3
\begin{frame}{Theoretical Justification}
\begin{itemize}
    \item Ensemble Error: depends on base learners' error and their correlation
    \item Key idea: uncorrelated models with decent accuracy help each other
\end{itemize}
\[
E = \rho \cdot \bar{e} + (1 - \rho) \cdot \bar{e}^2
\]
\begin{itemize}
    \item $\bar{e}$: average individual error
    \item $\rho$: average correlation between classifiers
\end{itemize}
\end{frame}
% -- Explain how diversity and average error together influence ensemble performance.

% Slide 4
\begin{frame}{Diversity in Ensembles}
\begin{itemize}
    \item Diversity is critical for ensemble success
    \item Common diversity measures:
    \begin{itemize}
        \item Q-statistic
        \item Correlation coefficient
        \item Disagreement measure
    \end{itemize}
    \item Trade-off: Accuracy vs. Diversity
\end{itemize}
\end{frame}
% -- Provide examples and visuals if needed. Diversity isn’t always easy to measure.

% Slide 5
\begin{frame}{Bagging (Bootstrap Aggregating)}
\begin{itemize}
    \item Train each model on a bootstrap sample
    \item Combine predictions (voting/averaging)
    \item Reduces variance
\end{itemize}
\[
\hat{f}_{bag}(x) = \frac{1}{T} \sum_{t=1}^{T} f_t(x)
\]
\begin{itemize}
    \item Example: Random Forests
\end{itemize}
\end{frame}
% -- Explain bootstrapping and its statistical effect.

% Slide 6
\begin{frame}{Boosting}
\begin{itemize}
    \item Models are trained sequentially
    \item Each model focuses on errors of previous ones
    \item Reduces bias
\end{itemize}
\[
F_m(x) = F_{m-1}(x) + \alpha_m h_m(x)
\]
\begin{itemize}
    \item AdaBoost: Changes sample weights
    \item Gradient Boosting: Fits gradient of loss function
\end{itemize}
\end{frame}
% -- Highlight margin theory and intuitive working of AdaBoost/GBM.

% Slide 7
\begin{frame}{Stacking}
\begin{itemize}
    \item Combine predictions using a meta-learner
    \item Level-0: Base learners
    \item Level-1: Meta model
    \item Trained on out-of-fold predictions
\end{itemize}
\end{frame}
% -- Stress the importance of using validation data to train meta-learner.

% Slide 8
\begin{frame}{Conditions for Effective Ensembles}
\begin{itemize}
    \item Base learners should:
    \begin{itemize}
        \item Be accurate (better than random)
        \item Be diverse (make different errors)
    \end{itemize}
    \item Independence isn't required — complementarity is enough
\end{itemize}
\end{frame}
% -- Use diagrams or examples if you present this live.

% Slide 9
\begin{frame}{Advanced: PAC Learning & Bounds}
\begin{itemize}
    \item PAC theory gives error bounds on learning
    \item Ensemble generalization error can be bounded:
    \[
    R(H) \leq \hat{R}(H) + \sqrt{\frac{VC(H) \log(n)}{n}}
    \]
    \item Boosting improves margins → better generalization
\end{itemize}
\end{frame}
% -- Introduce margin theory and generalization capacity.

% Slide 10
\begin{frame}{Pruning and Bayesian View}
\begin{itemize}
    \item Ensemble pruning removes weak or redundant learners
    \item Bayesian model averaging:
    \[
    p(y|x) = \sum_{i} p(y|x, h_i) p(h_i)
    \]
    \item Gives a probabilistic view of ensemble prediction
\end{itemize}
\end{frame}
% -- This can segue into ensemble optimization strategies.

% Slide 11
\begin{frame}{Practical Considerations}
\begin{itemize}
    \item Pros:
    \begin{itemize}
        \item Higher accuracy
        \item Robust to noise
    \end{itemize}
    \item Cons:
    \begin{itemize}
        \item Slower training/inference
        \item Reduced interpretability
    \end{itemize}
    \item Don’t ensemble identical models!
\end{itemize}
\end{frame}
% -- Discuss when ensembles might not be worth the trade-off.

% Slide 12
\begin{frame}{Case Studies & Tools}
\begin{itemize}
    \item Random Forest: Scikit-learn
    \item XGBoost, LightGBM: Efficient boosting libraries
    \item Deep Ensembles: Snapshot ensembles, SWA, MC Dropout
    \item Kaggle usage and industry adoption
\end{itemize}
\end{frame}
% -- Show real results or demo tools if possible.

% Slide 13
\begin{frame}{Summary and Key Takeaways}
\begin{itemize}
    \item Ensembles improve performance by combining weak learners
    \item Success depends on diversity and accuracy
    \item Bagging, Boosting, and Stacking are key methods
    \item Balance performance and complexity
\end{itemize}
\end{frame}


\begin{frame}[standout]
    Thank you!
\end{frame}
\end{document}\documentclass{beamer}
\usetheme{metropolis} % Clean modern look

\usepackage{amsmath, amsfonts}
\usepackage{graphicx}
\usepackage{booktabs}
\usepackage{hyperref}

\title{Ensemble Learning Theory}
\subtitle{}
\date{}

\begin{document}

% Title Slide
{
\setbeamertemplate{frame footer}{\href{https://github.com/abkafi1234/Datamining_MachineLearning_Class/tree/main/Class_Slides}{Report Error}}
\setbeamerfont{frame footer}{size=\tiny} 
\begin{frame}
    \titlepage
\end{frame}
}

% Outline Slide
\begin{frame}{Outline}
    \tableofcontents
\end{frame}

% Section Slide: Introduction
\section{Introduction}

\begin{frame}{Introduction to Ensemble Learning}
\begin{itemize}
    \item \textbf{Ensemble Learning} combines multiple models to improve performance.
    \item Motivation:
    \begin{itemize}
        \item Reduce overfitting
        \item Improve generalization
        \item Handle complex tasks
    \end{itemize}
    \item Widely used in practice: \textit{Random Forest}, \textit{XGBoost}, \textit{Stacking models}
\end{itemize}
\end{frame}

% Section Slide: Theory
\section{Theoretical Foundations}

\begin{frame}{Bias-Variance Decomposition}
\begin{itemize}
    \item Prediction Error = Bias$^2$ + Variance + Noise
    \item \textbf{Bagging}: reduces variance
    \item \textbf{Boosting}: reduces bias
\end{itemize}
\[
\text{Error} = \underbrace{\text{Bias}^2}_{\text{underfitting}} + \underbrace{\text{Variance}}_{\text{overfitting}} + \text{Noise}
\]
\end{frame}

\begin{frame}{Theoretical Justification}
\begin{itemize}
    \item Ensemble error depends on:
    \begin{itemize}
        \item Average error of base learners
        \item Correlation between them
    \end{itemize}
\end{itemize}
\[
E = \rho \cdot \bar{e} + (1 - \rho) \cdot \bar{e}^2
\]
\begin{itemize}
    \item $\bar{e}$: average individual error
    \item $\rho$: average pairwise correlation
\end{itemize}
\end{frame}

\begin{frame}{Diversity in Ensembles}
\begin{itemize}
    \item Key for success: learners should make different errors
    \item Measures:
    \begin{itemize}
        \item Q-statistic
        \item Correlation coefficient
        \item Disagreement measure
    \end{itemize}
    \item Too much similarity $\Rightarrow$ reduced ensemble benefit
\end{itemize}
\end{frame}

% Section Slide: Methods
\section{Major Ensemble Methods}

\begin{frame}{Overview of Ensemble Methods}
\begin{itemize}
    \item \textbf{Bagging}: Train in parallel on bootstrapped data
    \item \textbf{Boosting}: Train sequentially to fix errors
    \item \textbf{Stacking}: Combine diverse learners using a meta-learner
\end{itemize}
\end{frame}

\begin{frame}{Bagging (Bootstrap Aggregating)}
\begin{itemize}
    \item Learn $f_t(x)$ from bootstrapped sample of dataset
    \item Final prediction:
\[
\hat{f}_{bag}(x) = \frac{1}{T} \sum_{t=1}^{T} f_t(x)
\]
    \item Reduces variance without increasing bias
    \item Example: \textbf{Random Forest}
\end{itemize}
\end{frame}

\begin{frame}{Boosting}
\begin{itemize}
    \item Models trained sequentially
    \item Each new model focuses on previous errors
    \item Final prediction:
\[
F_m(x) = F_{m-1}(x) + \alpha_m h_m(x)
\]
    \item Types:
    \begin{itemize}
        \item \textbf{AdaBoost}: weighted majority voting
        \item \textbf{Gradient Boosting}: uses gradient of loss
    \end{itemize}
\end{itemize}
\end{frame}

\begin{frame}{Stacking (Stacked Generalization)}
\begin{itemize}
    \item Trains multiple diverse models (Level-0)
    \item Combine outputs via meta-learner (Level-1)
    \item Level-1 trained on out-of-fold predictions to avoid overfitting
    \item Improves robustness across tasks
\end{itemize}
\end{frame}

% Section Slide: Requirements & Analysis
\section{Conditions \& Theory}

\begin{frame}{Conditions for Effective Ensembles}
\begin{itemize}
    \item Base learners should:
    \begin{itemize}
        \item Be more accurate than random
        \item Make uncorrelated errors
    \end{itemize}
    \item Complementarity > independence
    \item Trade-off: more models vs better models
\end{itemize}
\end{frame}

\begin{frame}{PAC Learning and Generalization Bounds}
\begin{itemize}
    \item PAC theory gives sample complexity bounds
    \item Ensemble generalization error:
\[
R(H) \leq \hat{R}(H) + \sqrt{\frac{VC(H) \log(n)}{n}}
\]
    \item Boosting increases margin → better generalization
\end{itemize}
\end{frame}

% Section Slide: Practical Considerations
\section{Advanced Topics \& Practice}

\begin{frame}{Pruning and Bayesian View}
\begin{itemize}
    \item \textbf{Pruning}: Remove weak learners to reduce complexity
    \item \textbf{Bayesian Model Averaging (BMA)}:
\[
p(y|x) = \sum_i p(y|x, h_i) p(h_i)
\]
    \item Helps quantify uncertainty in ensemble predictions
\end{itemize}
\end{frame}

\begin{frame}{Practical Considerations}
\begin{itemize}
    \item \textbf{Pros}:
    \begin{itemize}
        \item Higher accuracy
        \item More robust predictions
    \end{itemize}
    \item \textbf{Cons}:
    \begin{itemize}
        \item Higher computation cost
        \item Less interpretability
    \end{itemize}
    \item Tip: Avoid using identical models
\end{itemize}
\end{frame}

\begin{frame}{Case Studies and Tools}
\begin{itemize}
    \item \textbf{Scikit-learn} (Random Forest, BaggingClassifier)
    \item \textbf{XGBoost}, \textbf{LightGBM} for boosting
    \item \textbf{Snapshot Ensembles}, \textbf{MC Dropout} in deep learning
    \item Widely used in Kaggle competitions and industry
\end{itemize}
\end{frame}

% Section Slide: Summary
\section{Summary}

\begin{frame}{Key Takeaways}
\begin{itemize}
    \item Ensemble methods improve performance by combining models
    \item Bagging reduces variance, Boosting reduces bias
    \item Diversity among models is crucial
    \item Be mindful of trade-offs: accuracy vs. complexity
\end{itemize}
\end{frame}

\begin{frame}[standout]
    Thank you!
\end{frame}

\end{document}
