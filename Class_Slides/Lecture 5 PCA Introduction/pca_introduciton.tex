\documentclass{beamer}
\usetheme{metropolis} % Clean modern look

\usepackage{amsmath, amsfonts}
\usepackage{graphicx}
\usepackage{booktabs}
\usepackage{hyperref}

\usepackage{listings}

\title{PCA Introduction}
\subtitle{}
\date{}

\definecolor{codegray}{gray}{0.95}
\lstset{
    backgroundcolor=\color{codegray},
    basicstyle=\ttfamily\small,
    breaklines=true,
    frame=single,
    showstringspaces=false
}

\begin{document}

% Title Slide
{
\setbeamertemplate{frame footer}{\href{https://github.com/abkafi1234/Datamining_MachineLearning_Class/tree/main/Class_Slides}{Report Error}}
\setbeamerfont{frame footer}{size=\tiny} 
\begin{frame}
    \titlepage
\end{frame}
}

% Outline Slide
\begin{frame}{Outline}
    \tableofcontents
\end{frame}

\section{Introduction}
\begin{frame}{What is PCA?}
\begin{itemize}
    \item Unsupervised dimensionality reduction technique
    \item Transforms original variables into uncorrelated \textbf{principal components}
    \item Captures the directions of highest variance
    \item Common uses: visualization, noise reduction, speeding up ML
\end{itemize}
\end{frame}

% Slide: When to Use PCA
\begin{frame}{When to Use PCA}
\begin{itemize}
    \item When you have \textbf{many features} and want to reduce complexity
    \item Features are \textbf{correlated}
    \item You want to \textbf{visualize} high-dimensional data
    \item To help \textbf{improve performance} or reduce overfitting
\end{itemize}
\end{frame}

% Slide: Key Concepts
\begin{frame}{Key Concepts}
\begin{itemize}
    \item \textbf{Variance:} Spread of data
    \item \textbf{Covariance matrix:} Relationship between features
    \item \textbf{Eigenvectors:} Principal directions
    \item \textbf{Eigenvalues:} Amount of variance explained
\end{itemize}
\end{frame}

% Slide: PCA Steps
\begin{frame}{PCA Step-by-Step}
\begin{enumerate}
    \item \textbf{Standardize} the data
    \item Compute \textbf{Covariance Matrix}
    \item Compute \textbf{Eigenvectors and Eigenvalues}
    \item Select top \textbf{k components}
    \item \textbf{Transform} the data
\end{enumerate}
\end{frame}

\section{The steps}
\begin{frame}{Step 1: Standardize the Data}
\small
\textbf{What:} \\
Center each feature (mean = 0), scale to unit variance. \\
\[
X_{scaled} = \frac{X - \mu}{\sigma}
\]

\vspace{1em}
\textbf{Why:} \\
PCA is scale-sensitive. Without standardization, features with larger magnitudes will dominate the principal components.
\end{frame}

% Slide 2: Compute Covariance Matrix
\begin{frame}{Step 2: Compute the Covariance Matrix}
\small
\textbf{What:} \\
Calculate the covariance between each pair of features. \\
\[
\mathbf{C} = \frac{1}{n - 1} X^\top X
\]

\vspace{1em}
\textbf{Why:} \\
The covariance matrix reveals the linear relationships between features — essential to find directions of greatest variance.
\end{frame}

% Slide 3: Compute Eigenvectors and Eigenvalues
\begin{frame}{Step 3: Compute Eigenvectors and Eigenvalues}
\small
\textbf{What:} \\
Solve the equation: \\
\[
\mathbf{C} \mathbf{w} = \lambda \mathbf{w}
\] 
Eigenvectors = directions; Eigenvalues = variance explained.

\vspace{1em}
\textbf{Why:} \\
To discover the axes (principal components) along which the data varies the most.
\end{frame}

% Slide 4: Select Top k Components
\begin{frame}{Step 4: Select Top \( k \) Components}
\small
\textbf{What:} \\
Rank eigenvalues in descending order and pick the top \( k \). \\
\[
\text{Retain the first } k \text{ eigenvectors}
\]

\vspace{1em}
\textbf{Why:} \\
To reduce dimensionality while preserving most of the variance (information) in the dataset.
\end{frame}

% Slide 5: Transform the Data
\begin{frame}{Step 5: Transform the Data}
\small
\textbf{What:} \\
Project the original data onto the new basis: \\
\[
Z = X \cdot W_k
\]

\vspace{1em}
\textbf{Why:} \\
This gives a new set of uncorrelated features in a lower-dimensional space — useful for visualization or model input.
\end{frame}

\section{Python Implementation}
% Slide: Python Code Example
\begin{frame}[fragile]{PCA in Python}
    \small
\begin{lstlisting}[language=Python]
from sklearn.decomposition import PCA
from sklearn.preprocessing import StandardScaler
from sklearn.datasets import load_iris
import matplotlib.pyplot as plt

data = load_iris()
X = StandardScaler().fit_transform(data.data)

pca = PCA(n_components=2)
X_pca = pca.fit_transform(X)

plt.scatter(X_pca[:,0], X_pca[:,1], c=data.target)
plt.xlabel("PC1")
plt.ylabel("PC2")
plt.title("PCA on Iris Dataset")
plt.show()
\end{lstlisting}
\end{frame}

% Slide: Choosing Components
\begin{frame}[fragile]{Choosing Number of Components}
    \small
\begin{lstlisting}[language=Python]
pca = PCA().fit(X)
plt.plot(np.cumsum(pca.explained_variance_ratio_))
plt.xlabel('Number of Components')
plt.ylabel('Cumulative Explained Variance')
plt.grid(True)
plt.show()
\end{lstlisting}
\end{frame}


% Slide: Pros and Cons
\begin{frame}{ Pros and  Cons}
\begin{columns}
\column{0.5\textwidth}
\textbf{Pros:}
\begin{itemize}
    \item Reduces overfitting
    \item Speeds up training
    \item Useful for visualization
    \item Removes multicollinearity
\end{itemize}

\column{0.5\textwidth}
\textbf{Cons:}
\begin{itemize}
    \item Loss of interpretability
    \item Assumes linearity
    \item Sensitive to feature scaling
    \item May discard useful info
\end{itemize}
\end{columns}
\end{frame}

% Slide: Applications
\begin{frame}{Applications of PCA}
\begin{itemize}
    \item Face recognition and image compression
    \item Noise filtering and denoising
    \item Gene expression analysis
    \item Financial modeling and portfolio risk
    \item Data visualization in 2D/3D
\end{itemize}
\end{frame}

% Slide: Summary
\begin{frame}{Summary}
\begin{itemize}
    \item PCA is a powerful tool for reducing dimensionality
    \item It transforms data into uncorrelated, variance-maximizing components
    \item Useful for improving model performance, visualizing data, and removing noise
\end{itemize}
\end{frame}

\begin{frame}[standout]
    Thank you!
\end{frame}
\end{document}
