\documentclass{beamer}
\usetheme{metropolis}

\usepackage{amsmath, amsfonts}
\usepackage{booktabs}
\usepackage{hyperref}

\title{Recurrent Neural Networks}
\subtitle{Understanding Sequential Models}
\date{July 19, 2025}

\begin{document}

{
\setbeamertemplate{frame footer}{\href{https://github.com/abkafi1234/Datamining_MachineLearning_Class/tree/main/Class_Slides}{Report Error}}
\setbeamerfont{frame footer}{size=\tiny}
\begin{frame}
    \titlepage
\end{frame}
}

\begin{frame}{Outline}
    \tableofcontents
\end{frame}

% --- Introduction ---
\section{What are RNNs?}
\begin{frame}{Why Recurrent Neural Networks?}
RNNs are designed to handle sequential data.
\begin{itemize}
    \item Traditional neural networks assume independent inputs
    \item RNNs capture dependencies across time steps
    \item Suitable for time series, language, speech, etc.
\end{itemize}
\end{frame}

\begin{frame}{Sequence Modeling}
Given a sequence \( x = [x_1, x_2, \dots, x_T] \), RNN learns a function:
\[
h_t = f(h_{t-1}, x_t)
\]
\begin{itemize}
    \item \( h_t \): hidden state at time \( t \)
    \item Encodes past information from sequence
\end{itemize}
\end{frame}

% --- RNN Mechanics ---
\section{RNN Mechanics}
\begin{frame}{Basic RNN Cell}
At each time step:
\[
h_t = \tanh(W_h h_{t-1} + W_x x_t + b)
\]
\[
y_t = W_y h_t + b_y
\]
\begin{itemize}
    \item \( x_t \in \mathbb{R}^{n_x} \): input vector
    \item \( h_t \in \mathbb{R}^{n_h} \): hidden state
    \item \( y_t \in \mathbb{R}^{n_y} \): output
\end{itemize}
\end{frame}

\begin{frame}{Shape Calculations}
Let:
\begin{itemize}
    \item Input shape: \( (T, n_x) \)
    \item Hidden state size: \( n_h \)
    \item Output size: \( n_y \)
\end{itemize}
Then:
\begin{itemize}
    \item \( W_x \in \mathbb{R}^{n_h \times n_x} \)
    \item \( W_h \in \mathbb{R}^{n_h \times n_h} \)
    \item \( W_y \in \mathbb{R}^{n_y \times n_h} \)
\end{itemize}
\end{frame}

% --- Learning and Backprop ---
\section{Training RNNs}
\begin{frame}{Backpropagation Through Time (BPTT)}
Gradients are computed over all time steps:
\[
\frac{\partial L}{\partial W} = \sum_{t=1}^T \frac{\partial L}{\partial h_t} \cdot \frac{\partial h_t}{\partial W}
\]
\begin{itemize}
    \item Computationally expensive for long sequences
    \item Suffers from vanishing/exploding gradients
\end{itemize}
\end{frame}

\begin{frame}{RNN Limitations}
\begin{itemize}
    \item \textbf{Vanishing gradients}: long-term dependencies hard to learn
    \item \textbf{Exploding gradients}: unstable training
    \item \textbf{Short memory}: can't retain info over long sequences
\end{itemize}
\end{frame}

% --- Variants ---
\section{Variants of RNNs}
\begin{frame}{Variants of RNNs}
\begin{itemize}
    \item \textbf{LSTM (Long Short-Term Memory)}:
        \begin{itemize}
            \item Uses gates to control information flow
            \item Remembers information over longer periods
        \end{itemize}
    \item \textbf{GRU (Gated Recurrent Unit)}:
        \begin{itemize}
            \item Simplified LSTM
            \item Fewer parameters, faster training
        \end{itemize}
\end{itemize}
\end{frame}

\begin{frame}{When to Use LSTM or GRU}
\begin{itemize}
    \item Use LSTM for long sequences with complex dependencies
    \item Use GRU for faster training with decent performance
    \item Start simple, benchmark both
\end{itemize}
\end{frame}

% --- Applications ---
\section{Applications}
\begin{frame}{Applications of RNNs}
\begin{itemize}
    \item Language modeling
    \item Machine translation
    \item Speech recognition
    \item Music generation
    \item Time series forecasting
\end{itemize}
\end{frame}

% --- Summary ---
\section{Summary}
\begin{frame}{Key Concepts Recap}
\begin{itemize}
    \item RNNs process sequences step-by-step
    \item Hidden states capture memory
    \item Training uses backpropagation through time
    \item LSTM and GRU address gradient issues
\end{itemize}
\end{frame}

\begin{frame}[standout]
    Thank you! \\
    Questions?
\end{frame}

\end{document}
