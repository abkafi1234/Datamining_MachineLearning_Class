\documentclass{beamer}
\usetheme{metropolis} % Clean modern look

% \usepackage{fontspec} % Needed for Unicode, if using emojis
\usepackage{amsmath, amsfonts}
\usepackage{graphicx}
\usepackage{booktabs}
\usepackage{hyperref}

\title{Measures of Distance in Data Mining }
\subtitle{How similar are two data points?}
% \author{Your Name}
\date{}

\begin{document}
{
\setbeamertemplate{frame footer}{\href{https://github.com/abkafi1234/Datamining_MachineLearning_Class/tree/main/Class_Slides}{Report Error}}
\setbeamerfont{frame footer}{size=\tiny} 
\begin{frame}
    \titlepage
\end{frame}
}
\begin{frame}{Outline}
    \tableofcontents
\end{frame}


% --- Why Measure Distance ---
\section{Why Measure Distance?}
\begin{frame}{Why Measure Distance? (Theory)}
\textbf{Distance measures} quantify the similarity or dissimilarity between data points. They are foundational in:
\begin{itemize}
    \item Clustering (e.g., K-Means, Hierarchical)
    \item Classification (e.g., K-Nearest Neighbors)
    \item Anomaly detection
    \item Information retrieval and recommender systems
\end{itemize}
Choosing the right distance measure affects model accuracy and interpretability.
\end{frame}

% === EUCLIDEAN DISTANCE ===
\section{Euclidean Distance}
\begin{frame}{Euclidean Distance (Theory)}
\textbf{Euclidean distance} is the most common measure of straight-line distance in continuous space.
\begin{itemize}
    \item Sensitive to magnitude and scale
    \item Assumes continuous, real-valued attributes
\end{itemize}
\end{frame}

\begin{frame}{Euclidean Distance (Math)}
Given two points \( A = (x_1, x_2, \dots, x_n) \) and \( B = (y_1, y_2, \dots, y_n) \), the Euclidean distance is:
\[
d(A, B) = \sqrt{\sum_{i=1}^{n}(x_i - y_i)^2}
\]
\end{frame}

\begin{frame}{Euclidean Distance (Example)}
Given \( A = (2, 4) \), \( B = (5, 8) \):

\[
d(A, B) = \sqrt{(2 - 5)^2 + (4 - 8)^2}
= \sqrt{(-3)^2 + (-4)^2} = \sqrt{9 + 16} = \sqrt{25} = 5
\]
\end{frame}

\begin{frame}{Euclidean Distance (Use Case)}
\textbf{Use Case:} K-Means Clustering
\begin{itemize}
    \item Used to assign points to the nearest centroid
    \item Ideal when clusters are spherical and scales are normalized
\end{itemize}
\end{frame}

% === MANHATTAN DISTANCE ===
\section{Manhattan Distance}
\begin{frame}{Manhattan Distance (Theory)}
Also called **Taxicab** or **City Block Distance**:
\begin{itemize}
    \item Measures distance by summing absolute differences
    \item Better for high-dimensional, sparse data
\end{itemize}
\end{frame}

\begin{frame}{Manhattan Distance (Math)}
\[
d(A, B) = \sum_{i=1}^{n} |x_i - y_i|
\]
Where \( A = (x_1, \dots, x_n) \), \( B = (y_1, \dots, y_n) \)
\end{frame}
\begin{frame}{Manhattan Distance (Example)}
Given \( A = (2, 4) \), \( B = (5, 8) \):

\[
d(A, B) = |2 - 5| + |4 - 8| = 3 + 4 = 7
\]
\end{frame}

\begin{frame}{Manhattan Distance (Use Case)}
\textbf{Use Case:} L1-Regularized models (e.g., Lasso Regression)
\begin{itemize}
    \item Promotes sparsity
    \item Preferred when feature differences are linear or additive
\end{itemize}
\end{frame}

% === JACCARD INDEX ===
\section{Jaccard Index}
\begin{frame}{Jaccard Index (Theory)}
\textbf{Jaccard Similarity} measures overlap between sets.
\begin{itemize}
    \item Works with binary or categorical data
    \item Good for market basket analysis, document similarity
\end{itemize}
\end{frame}

\begin{frame}{Jaccard Index (Math)}
\[
J(A, B) = \frac{|A \cap B|}{|A \cup B|}
\]
\[
d(A, B) = 1 - J(A, B)
\]
\end{frame}
\begin{frame}{Jaccard Index (Example)}
Let \( A = \{1, 2, 3, 5\} \), \( B = \{2, 3, 4, 6\} \)

\[
A \cap B = \{2, 3\},\quad A \cup B = \{1, 2, 3, 4, 5, 6\}
\]

\[
J(A, B) = \frac{2}{6} = 0.33,\quad d(A, B) = 1 - 0.33 = 0.67
\]
\end{frame}

\begin{frame}{Jaccard Index (Use Case)}
\textbf{Use Case:} Recommender Systems
\begin{itemize}
    \item Used to compare user interests (e.g., liked items)
    \item Suitable for sparse, binary user-item matrices
\end{itemize}
\end{frame}

% === MINKOWSKI DISTANCE ===
\section{Minkowski Distance}
\begin{frame}{Minkowski Distance (Theory)}
Generalized distance metric that includes:
\begin{itemize}
    \item Euclidean distance when \( p = 2 \)
    \item Manhattan distance when \( p = 1 \)
\end{itemize}
\end{frame}

\begin{frame}{Minkowski Distance (Math)}
\[
d(A, B) = \left( \sum_{i=1}^{n} |x_i - y_i|^p \right)^{1/p}
\]
Where \( p \in \mathbb{R},\ p \geq 1 \)
\end{frame}
\begin{frame}{Minkowski Distance (Example)}
Given \( A = (1, 2) \), \( B = (4, 6) \), with \( p = 3 \):

\[
d(A, B) = \left(|1 - 4|^3 + |2 - 6|^3\right)^{1/3}
= \left(27 + 64\right)^{1/3} = (91)^{1/3} \approx 4.481
\]
\end{frame}

\begin{frame}{Minkowski Distance (Use Case)}
\textbf{Use Case:} Customizable distance metric in KNN
\begin{itemize}
    \item Choose \( p \) based on desired sensitivity
    \item Flexible for tuning similarity in various data distributions
\end{itemize}
\end{frame}

% === COSINE SIMILARITY ===
\section{Cosine Similarity}
\begin{frame}{Cosine Similarity (Theory)}
\textbf{Cosine Similarity} measures the cosine of the angle between two vectors.
\begin{itemize}
    \item Focuses on orientation, not magnitude
    \item Ideal for high-dimensional, sparse data like text
\end{itemize}
\end{frame}

\begin{frame}{Cosine Similarity (Math)}
\[
\text{Cosine}(A, B) = \frac{A \cdot B}{\|A\| \cdot \|B\|} = \frac{\sum_{i=1}^{n} x_i y_i}{\sqrt{\sum x_i^2} \cdot \sqrt{\sum y_i^2}}
\]
\end{frame}
\begin{frame}{Cosine Similarity (Example)}
\textbf{Example 1: Similar Vectors}

Let \( A_1 = (1, 2, 3) \), \( B_1 = (2, 4, 6) \)

\[
\cos(\theta) = \frac{1\cdot2 + 2\cdot4 + 3\cdot6}{\sqrt{1^2+2^2+3^2} \cdot \sqrt{2^2+4^2+6^2}} = \frac{28}{\sqrt{14} \cdot \sqrt{56}} = 1
\]

\textit{Perfect similarity — linearly dependent or colinear (same direction)}

\vspace{1em}

\textbf{Example 2: Dissimilar Vectors}

Let \( A_2 = (1, 0) \), \( B_2 = (0, 1) \)

\[
\cos(\theta) = \frac{1\cdot0 + 0\cdot1}{\sqrt{1^2+0^2} \cdot \sqrt{0^2+1^2}} = 0
\]

\textit{No similarity — orthogonal vectors.}
\end{frame}


\begin{frame}{Cosine Similarity (Use Case)}
\textbf{Use Case:} Document Similarity in NLP
\begin{itemize}
    \item Used to compare TF-IDF or word embeddings
    \item Common in search engines, chatbots, and plagiarism detection
\end{itemize}
\end{frame}

% --- Final Summary ---
\begin{frame}{Summary Table}
\footnotesize
\begin{tabular}{@{}llll@{}}
\toprule
\textbf{Metric} & \textbf{Data Type} & \textbf{Math Form} & \textbf{Use Case} \\
\midrule
Euclidean & Numeric & \( \sqrt{\sum (x_i - y_i)^2} \) & K-Means Clustering \\
Manhattan & Numeric & \( \sum |x_i - y_i| \) & Lasso Regression \\
Jaccard & Set/Binary & \( \frac{|A \cap B|}{|A \cup B|} \) & Recommender Systems \\
Minkowski & Numeric & \( (\sum |x_i - y_i|^p)^{1/p} \) & Custom KNN \\
Cosine & Vector/High-dim & \( \frac{A \cdot B}{\|A\| \|B\|} \) & Text Similarity \\
\bottomrule
\end{tabular}
\end{frame}

\section{Some More Distance Measures}
\begin{frame}{What is Mahalanobis Distance?}
  \begin{itemize}
    \item Measures distance between a point and a distribution
    \item Accounts for correlation between variables
    \item Scale-invariant
    \item More robust than Euclidean distance for multivariate data
  \end{itemize}
\end{frame}

\begin{frame}{Intuition}
  \begin{itemize}
    \item Euclidean distance treats all variables equally
    \item Mahalanobis considers variable correlation and scale
    \item Tells how many standard deviations a point is from the mean
    \item Adapts to the shape and orientation of the data
  \end{itemize}
\end{frame}
\begin{frame}{Mathematical Formulation}
  Given:
  \begin{itemize}
    \item Point: \( \mathbf{x} \)
    \item Mean: \( \boldsymbol{\mu} \)
    \item Covariance: \( \Sigma \)
  \end{itemize}
  \vspace{0.5em}
  \[
    D_M(x) = \sqrt{(\mathbf{x} - \boldsymbol{\mu})^T \Sigma^{-1} (\mathbf{x} - \boldsymbol{\mu})}
  \]
  \begin{itemize}
    \item \( \Sigma^{-1} \): Inverse of covariance matrix
    \item Output is a scalar distance
  \end{itemize}
\end{frame}
\begin{frame}{Properties}
  \begin{itemize}
    \item Accounts for scale and correlation of features
    \item Reduces to Euclidean distance if \( \Sigma = I \)
    \item \( D_M(x) = 0 \): Point at mean
    \item \( D_M(x) = 1 \): One standard deviation away
    \item Useful for identifying multivariate outliers
  \end{itemize}
\end{frame}
\begin{frame}{Applications}
  \begin{itemize}
    \item Outlier detection
    \item Clustering (e.g., GMMs)
    \item Multivariate hypothesis testing
    \item Pattern recognition (e.g., face recognition)
    \item Quality control in industrial settings
  \end{itemize}
\end{frame}

\begin{frame}{Example Calculation}
  \begin{itemize}
    \item Mean: \( \boldsymbol{\mu} = \begin{bmatrix} 5 \\ 3 \end{bmatrix} \)
    \item Covariance: \( \Sigma = \begin{bmatrix} 4 & 2 \\ 2 & 3 \end{bmatrix} \)
    \item Point: \( \mathbf{x} = \begin{bmatrix} 6 \\ 4 \end{bmatrix} \)
  \end{itemize}
  \vspace{0.5em}
  \[
    D_M(x) = \sqrt{(x - \mu)^T \Sigma^{-1} (x - \mu)}
  \]
\end{frame}
\begin{frame}[fragile]{Python Implementation}
    \small
\begin{verbatim}
import numpy as np
from scipy.spatial import distance

data = np.array([[2, 3], [3, 5], [4, 8], [5, 11]])
x = np.array([3, 7])
mean = np.mean(data, axis=0)
cov = np.cov(data, rowvar=False)
inv_cov = np.linalg.inv(cov)

# Manual calculation
diff = x - mean
d_mahal = np.sqrt(diff.T @ inv_cov @ diff)

# Using scipy
d_scipy = distance.mahalanobis(x, mean, inv_cov)
\end{verbatim}
\end{frame}

\begin{frame}{Tips \& Best Practices}
  \begin{itemize}
    \item Ensure data follows approximately multivariate normal distribution
    \item Handle singular covariance matrices (regularization, PCA)
    \item Normalize data if using with other distance metrics
    \item Be cautious with high-dimensional data (curse of dimensionality)
  \end{itemize}
\end{frame}


\begin{frame}{What is MCC?}
  \begin{itemize}
    \item Metric for evaluating binary (and multiclass) classification
    \item Considers: TP, TN, FP, FN
    \item Balanced measure, even with class imbalance
    \item Interpreted as a correlation coefficient
  \end{itemize}
\end{frame}


\begin{frame}{Intuition}
  \begin{itemize}
    \item Accuracy fails in imbalanced datasets
    \item MCC reflects the relationship between predictions and ground truth
    \item Robust and symmetric
  \end{itemize}
  \vspace{0.5em}
  \textbf{Values:}
  \begin{itemize}
    \item 1 = Perfect prediction
    \item 0 = No better than random
    \item -1 = Total disagreement
  \end{itemize}
\end{frame}

\begin{frame}{MCC Formula}
  \[
    \text{MCC} = \frac{(TP \cdot TN) - (FP \cdot FN)}{
    \sqrt{(TP + FP)(TP + FN)(TN + FP)(TN + FN)}}
  \]
  \begin{itemize}
    \item If denominator = 0, MCC is defined as 0
    \item Normalized between -1 and 1
  \end{itemize}
\end{frame}

\begin{frame}{Properties}
  \begin{itemize}
    \item \(-1 \leq \text{MCC} \leq 1\)
    \item Handles class imbalance
    \item Invariant to label flipping
    \item More reliable than accuracy or F1-score alone
  \end{itemize}
\end{frame}

\begin{frame}{Applications}
  \begin{itemize}
    \item Medical diagnostics
    \item Fraud detection
    \item Binary classifiers in NLP, vision
    \item Model selection and benchmarking
    \item Extended to multiclass and multilabel problems
  \end{itemize}
\end{frame}

\begin{frame}{Example Calculation}
  \textbf{Confusion Matrix:}
  \begin{itemize}
    \item TP = 70, TN = 50
    \item FP = 10, FN = 5
  \end{itemize}
  \[
    \text{MCC} = \frac{(70 \cdot 50) - (10 \cdot 5)}{
    \sqrt{(70+10)(70+5)(50+10)(50+5)}}
  \]
  \[
    = \frac{3450}{\sqrt{19800000}} \approx 0.776
  \]
\end{frame}

\begin{frame}[fragile]{Python Implementation}
\begin{verbatim}
from sklearn.metrics import matthews_corrcoef

y_true = [1, 1, 0, 1, 0, 0, 1]
y_pred = [1, 0, 0, 1, 0, 1, 1]

mcc = matthews_corrcoef(y_true, y_pred)
print("MCC:", mcc)
\end{verbatim}
\end{frame}

% Slide 8: Manual Python
\begin{frame}[fragile]{Manual Calculation}
    \small
\begin{verbatim}
from math import sqrt

TP, TN, FP, FN = 70, 50, 10, 5
numerator = (TP * TN) - (FP * FN)
denominator = sqrt((TP + FP)*(TP + FN)*(TN + FP)*(TN + FN))
mcc = numerator / denominator if denominator != 0 else 0
print("MCC (manual):", mcc)
\end{verbatim}
\end{frame}

\begin{frame}{Tips \& Best Practices}
  \begin{itemize}
    \item Use MCC for imbalanced datasets
    \item Report MCC with other metrics (e.g., AUC, F1)
    \item Useful in production to track model drift
    \item Preferable when false positives/negatives carry different costs
  \end{itemize}
\end{frame}

\begin{frame}[standout]
    Thank you!
\end{frame}
\end{document}