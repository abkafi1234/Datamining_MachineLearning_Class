\documentclass{beamer}
\usetheme{metropolis}

\usepackage{amsmath, amsfonts}
\usepackage{booktabs}
\usepackage{hyperref}

\title{Trustworthy AI}
\subtitle{Fairness, Transparency, and Robustness}
\date{July 19, 2025}

\begin{document}

{
\setbeamertemplate{frame footer}{\href{https://github.com/abkafi1234/Datamining_MachineLearning_Class/tree/main/Class_Slides}{Report Error}}
\setbeamerfont{frame footer}{size=\tiny}
\begin{frame}
    \titlepage
\end{frame}
}

\begin{frame}{Outline}
    \tableofcontents
\end{frame}

% --- Intro ---
\section{Introduction}
\begin{frame}{Why Trustworthy AI?}
As AI becomes widespread, we must ensure it is:
\begin{itemize}
    \item Fair — avoids discrimination
    \item Transparent — understandable and explainable
    \item Robust — secure and resilient to failures
\end{itemize}
\end{frame}

\begin{frame}{Core Pillars}
Trustworthy AI is guided by 3 major principles:
\begin{enumerate}
    \item \textbf{Fairness}
    \item \textbf{Transparency (Explainability)}
    \item \textbf{Robustness (Safety \& Security)}
\end{enumerate}
\end{frame}

% --- Fairness ---
\section{Fairness}
\begin{frame}{What is Fairness in AI?}
An AI system is fair if it does not produce discriminatory outcomes against individuals or groups based on:
\begin{itemize}
    \item Race, gender, age, religion
    \item Sensitive or protected attributes
\end{itemize}
\end{frame}

\begin{frame}{Fairness Metrics}
Let \( A \) be a sensitive attribute and \( \hat{Y} \) the predicted label.

\textbf{Demographic Parity:}
\[
P(\hat{Y} = 1 | A = 0) = P(\hat{Y} = 1 | A = 1)
\]

\textbf{Equalized Odds:}
\[
P(\hat{Y} = 1 | A = a, Y = y) \text{ should be equal for all } a
\]

\textbf{Individual Fairness:}
\[
\text{Similar individuals } \Rightarrow \text{ similar predictions}
\]
\end{frame}

\begin{frame}{Fairness Trade-offs}
\begin{itemize}
    \item Trade-offs exist between fairness and accuracy
    \item Achieving all fairness metrics at once is often impossible
    \item Context-specific fairness definitions must be selected
\end{itemize}
\end{frame}

% --- Transparency ---
\section{Transparency}
\begin{frame}{Why Transparency?}
\begin{itemize}
    \item Stakeholders must understand how AI makes decisions
    \item Lack of transparency reduces trust
    \item Enables accountability and debugging
\end{itemize}
\end{frame}

\begin{frame}{Types of Explainability}
\begin{itemize}
    \item \textbf{Global explanations}: model-level understanding
    \item \textbf{Local explanations}: explain individual predictions
    \item \textbf{Post-hoc}: use external tools (e.g., LIME, SHAP)
    \item \textbf{Interpretable models}: inherently simple (e.g., decision trees)
\end{itemize}
\end{frame}

\begin{frame}{Interpretable vs Black-box Models}
\footnotesize
\begin{tabular}{@{}lll@{}}
\toprule
\textbf{Model Type} & \textbf{Interpretability} & \textbf{Examples} \\
\midrule
Linear Models & High & Logistic Regression \\
Decision Trees & High & CART, C4.5 \\
Neural Nets & Low & CNNs, RNNs \\
Ensembles & Low & Random Forest, XGBoost \\
\bottomrule
\end{tabular}
\end{frame}

% --- Robustness ---
\section{Robustness}
\begin{frame}{What is Robustness?}
\textbf{Robustness} refers to an AI system's ability to remain:
\begin{itemize}
    \item Accurate under noisy, adversarial, or out-of-distribution inputs
    \item Stable when small changes are made to input
    \item Resilient against manipulation and attacks
\end{itemize}
\end{frame}

\begin{frame}{Types of Robustness}
\begin{itemize}
    \item \textbf{Adversarial robustness} — resisting malicious perturbations
    \item \textbf{Distributional shift} — generalizing to new data domains
    \item \textbf{Resilience to missing/noisy data}
\end{itemize}
\end{frame}

\begin{frame}{Simple Adversarial Example}
Let \( x \) be an image input and \( \hat{y} = f(x) \).  
We add perturbation \( \delta \), where \( \|\delta\| \) is small:

\[
\hat{y} = f(x), \quad \hat{y}' = f(x + \delta), \quad \hat{y} \ne \hat{y}'
\]

\begin{itemize}
    \item Model prediction flips even with imperceptible change
    \item Needs robust training to defend
\end{itemize}
\end{frame}

% --- Best Practices ---
\section{Best Practices}
\begin{frame}{Best Practices for Trustworthy AI}
\begin{itemize}
    \item Use interpretable models when possible
    \item Audit datasets for bias
    \item Evaluate fairness using metrics
    \item Perform robustness testing (e.g., adversarial attacks)
    \item Document model behavior clearly
\end{itemize}
\end{frame}

% --- Summary ---
\begin{frame}{Key Takeaways}
\begin{itemize}
    \item Trustworthy AI ensures fairness, transparency, robustness
    \item Must balance ethical principles with performance
    \item Requires tools, metrics, and careful design
\end{itemize}
\end{frame}

\begin{frame}[standout]
    Thank you! \\
    Questions?
\end{frame}

\end{document}
