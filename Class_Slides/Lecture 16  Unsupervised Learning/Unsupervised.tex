\documentclass{beamer}
\usetheme{metropolis}

\usepackage{amsmath, amsfonts}
\usepackage{graphicx}
\usepackage{booktabs}
\usepackage{hyperref}

\title{Unsupervised Learning}
\subtitle{A Beginner's Introduction}
\date{}

\begin{document}

% Title Slide
{
\setbeamertemplate{frame footer}{\href{https://github.com/abkafi1234/Datamining_MachineLearning_Class/tree/main/Class_Slides}{Report Error}}
\setbeamerfont{frame footer}{size=\tiny}
\begin{frame}
    \titlepage
\end{frame}
}

% Outline
\begin{frame}{Outline}
    \tableofcontents
\end{frame}

% Section 1
\section{Introduction}

\begin{frame}{What is Unsupervised Learning?}
\begin{itemize}
    \item Learning patterns from \textbf{unlabeled} data
    \item No predefined output or target variable
    \item Goal: discover hidden structures, groupings, or representations
    \item Common in exploratory data analysis
\end{itemize}
\end{frame}

\begin{frame}{Why Unsupervised Learning?}
\begin{itemize}
    \item Labeling data is expensive or impossible
    \item Understand data distribution and relationships
    \item Useful for:
    \begin{itemize}
        \item Clustering customers, documents, images
        \item Reducing dimensionality for visualization or speed
        \item Detecting anomalies or outliers
    \end{itemize}
\end{itemize}
\end{frame}

% Section 2
\section{Main Tasks}

\begin{frame}{Clustering}
\begin{itemize}
    \item Group similar data points into clusters
    \item Examples:
    \begin{itemize}
        \item \textbf{k-Means}: partition data into k groups by minimizing within-cluster variance
        \item \textbf{Hierarchical Clustering}: build a tree of clusters
        \item \textbf{DBSCAN}: density-based clustering, detects noise
    \end{itemize}
\end{itemize}
\end{frame}

\begin{frame}{Dimensionality Reduction}
\begin{itemize}
    \item Reduce number of features while preserving structure
    \item Common methods:
    \begin{itemize}
        \item \textbf{PCA} (Principal Component Analysis): finds directions of maximum variance
        \item \textbf{t-SNE}, \textbf{UMAP}: nonlinear embeddings for visualization
        \item \textbf{Autoencoders}: neural networks that learn compressed representations
    \end{itemize}
\end{itemize}
\end{frame}

\begin{frame}{Density Estimation and Anomaly Detection}
\begin{itemize}
    \item Estimate probability distribution of data
    \item Detect outliers as points in low-density regions
    \item Examples:
    \begin{itemize}
        \item Gaussian Mixture Models (GMM)
        \item Kernel Density Estimation (KDE)
        \item One-class SVM
    \end{itemize}
\end{itemize}
\end{frame}

% Section 3
\section{Popular Algorithms}

\begin{frame}{k-Means Clustering}
\begin{itemize}
    \item Initialize k centroids randomly
    \item Assign points to nearest centroid
    \item Update centroids as mean of assigned points
    \item Repeat until convergence
\end{itemize}
\end{frame}

\begin{frame}{Principal Component Analysis (PCA)}
\begin{itemize}
    \item Linear projection to lower dimension
    \item Finds orthogonal directions maximizing variance
    \item Useful for:
    \begin{itemize}
        \item Noise reduction
        \item Visualization
        \item Feature extraction
    \end{itemize}
\end{itemize}
\end{frame}

% Section 4
\section{Challenges and Considerations}

\begin{frame}{Challenges in Unsupervised Learning}
\begin{itemize}
    \item No ground truth for evaluation
    \item Choosing number of clusters or components
    \item Sensitivity to initialization and parameters
    \item Scalability to large datasets
    \item Interpretability of results
\end{itemize}
\end{frame}

% Section 5
\section{Applications and Tools}

\begin{frame}{Applications}
\begin{itemize}
    \item Customer segmentation
    \item Document and image organization
    \item Anomaly detection in fraud, network security
    \item Data compression and visualization
\end{itemize}
\end{frame}

\begin{frame}{Popular Libraries}
\begin{itemize}
    \item \textbf{Scikit-learn}: clustering, PCA, GMM, DBSCAN
    \item \textbf{TensorFlow/PyTorch}: autoencoders and deep clustering
    \item \textbf{HDBSCAN}, \textbf{UMAP} packages for advanced clustering and visualization
\end{itemize}
\end{frame}

% Summary
\section{Summary}

\begin{frame}{Key Takeaways}
\begin{itemize}
    \item Unsupervised learning finds patterns without labels
    \item Key tasks: clustering, dimensionality reduction, density estimation
    \item Many algorithms exist; choice depends on data and goal
    \item Evaluation is often subjective or uses proxy metrics
\end{itemize}
\end{frame}

\begin{frame}[standout]
    Thank you!
\end{frame}

\end{document}
